\documentclass{resume}

\usepackage{hyperref}
\usepackage[left=0.70in,top=0.5in,right=0.70in,bottom=0.5in]{geometry}

\name{Andrea Tulimiero} % Your name
\address{111 Uetlibergstrasse, 8045$~\cdot~$Zurich, Switzerland$~\cdot~$\texttt{+}39\texttt{-}3400693725} % Your address
\address{tulimiero.andrea@gmail.com$~\cdot~$\href{https://github.com/andreatulimiero}{github.com/andreatulimiero}} % Your phone number and email

\begin{document}

%----------------------------------------------------------------------------------------
%	EDUCATION SECTION
%----------------------------------------------------------------------------------------

\begin{rSection}{Education}

{\bf ETH, Zurich} \hfill {\em September 2018 - August 2020} \\ 
M.Sc. in Computer Science\\
Partial GPA: 5.4 (out of 6.0)

{\bf Sapienza University of Rome, Rome} \hfill {\em September 2015 - June 2018} \\ 
B.Sc in Engineering in Computer Science\\
Final grade: 110 cum laude (out of 110)

\end{rSection}

%----------------------------------------------------------------------------------------
%	EXPERIENCE SECTION
%----------------------------------------------------------------------------------------

\begin{rSection}{Experience}

  \begin{rSubsection}{Software Engineer \& Research Assistant}{July 2019 - Present}{Network Security Group, ETH}{Zurich, Switzerland}
  % \item Analyzed security guarantees achieved by the \href{https://www.scion-architecture.net/}{SCION} next-generation internet architecture when targeted by DDoS attacks.
  \item Increased SCION Web App user engagement by 13\% by redesigning Database and API to include a feature to experiment with network multipath technology.  [\textbf{Python} (Django), HTML/\textbf{JavaScript}]
  \item Built a SCION compatible home router (SCION Box) to allow for the widespread adoption of the SCION internet architecture. [\textbf{OpenWRT}, Linux \textbf{net-tools}]
  \end{rSubsection}
  \begin{rSubsection}{Algorithms Laboratory (ETH)}{September 2019 - January 2020}{}{}
  \item Course on solving competitive programming problems by implementing a combination of basic (Greedy, Dynamic Programming, Graphs) and advanced (Network Flow) algorithms. [\textbf{C\texttt{++}}]
  \end{rSubsection}
  \begin{rSubsection}{Advanced Operating Systems (ETH)}{March 2019 - June 2019}{}{}
  \item Developed a research-oriented microkernel in a team of 3.
  \item Implemented physical and virtual memory manager, inter-process communication, and filesystem. [\textbf{C}]
  \end{rSubsection}
  \begin{rSubsection}{Google Workshop for Cloud and Development}{November 2017 - June 2018}{Sapienza University/Google Tel Aviv}{Rome, Italy/Tel Aviv, Israel}
  \item Coordinated a team of 5 to build a Web platform to drastically ease and speed up the sharing of findings between the malware researchers community. [\textbf{Python}, HTML/\textbf{JavaScript}]
  \end{rSubsection}

\end{rSection}

%----------------------------------------------------------------------------------------
%	LEADERSHIP AND AWARDS
%----------------------------------------------------------------------------------------

\begin{rSection}{Leadership and Awards}

\begin{rSubsection}{}{}{}{}
\item Speaker at the SCION Day conference with investors presenting the SCION Box. (talk available \href{https://video.ethz.ch/events/2019/scion/61dd8a87-3894-489d-99d3-77ca66c5ad38.html}{here})
\item Administrator of the ETH CTF team's capture the flag event.
\item Selected as one of 5 participants of Sapienza University's Honors Program for a semester-long research project about dynamic malware analysis. [\textbf{C\texttt{++}}, Intel \textbf{Pin}]
\item Awarded with Mario Negri scholarship for outstanding results achieved during my B.Sc. studies.
\end{rSubsection}

\end{rSection}

%----------------------------------------------------------------------------------------
%	TECHNICAL STRENGTHS SECTION
%----------------------------------------------------------------------------------------

\begin{rSection}{Programming Languages and Computer Science}
Python, C, C\texttt{++}, JavaScript, SQL \hfill Linux, TCP/IP, Data structures, Hadoop
\end{rSection}

%----------------------------------------------------------------------------------------
%	EXTRACURRICULAR ACTIVITIES
%----------------------------------------------------------------------------------------

\begin{rSection}{Extracurricular Activities}

\begin{rSubsection}{}{}{}{}
\item Competitive programmer on HackerRank and LeetCode. [\textbf{C}\texttt{++}]
\item Freelancer full-stack developer realizing ad-hoc solutions for small companies. [Django, Android]
\end{rSubsection}
\end{rSection}

\end{document}
